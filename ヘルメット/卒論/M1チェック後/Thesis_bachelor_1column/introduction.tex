\chapter{はじめに}
\label{introduction}
近年,販売されている四輪車の多くにスマートキーシステムを導入している.スマートキーシステムとは,電子キーをポケットなどに入れて接近すると電波を感知しドアの施錠や解錠,エンジンの始動をボタン一つで操作できるようにする機能である.このシステムは二輪車においても導入されつつある.二輪車におけるスマートキーシステムも同様に,電子キーを所持しておくことで,キーシリンダーにキーを挿し込む手間なくラゲッジスペースと呼ばれる荷物を収納するスペースの解錠やエンジンの始動が可能になる機能である.しかしながら,このシステムではキーを所持しておかなければならず,キーを紛失する恐れやキーの盗難による車両盗難のリスクがある.\par
本研究では,圧力センサを搭載したヘルメットを装着することで,頭部形状に基づき個人識別をする手法を提案する.二輪車での走行で必要であるヘルメットを用いた本人認証が実現できれば,既存のキーに関する問題点を解決できると考えられる.予め車両とリンクしておいたヘルメットをキーの代わりとして二輪車のヘルメットロックに装備しておく.そうすれば,利用者は身体一つで乗車できるため,キーの紛失のリスクを減少させることができる.利用者はヘルメットを被ることで個人認証をして,車両の持ち主がヘルメットを被った場合はエンジンの始動を可能にする.一方で,車両の持ち主以外がヘルメットを被った場合はエンジンの始動ができないようにすることで,車両盗難のリスクも減少させることができる.\par
個人認証の要素には各個人ごとに特徴があり,かつ複製が難しい身体部位を用いる必要がある.当麻ら\cite{face}が提案しているステレオカメラを用いた顔認証システムでは,少数のカメラで顔認証が行える方法が提案されている.顔認証は二輪車にあらかじめ取り付けておいたカメラを用いて,ヘルメット装着前にカメラの方を向くことで実装可能だと考えられる.しかし,カメラに対して屋外の過酷な環境への耐久性が求められる.近年,普及されつつある個人認証方法として指紋認証があげられる.指紋認証をヘルメットに搭載することも可能であるが,越前ら\cite{finger_print}の研究のように,指紋は写真などから簡単に複製されてしまうリスクを持つ.本研究で用いる頭部形状は人により異なる特徴が存在する.また,身体部位としても大きいため,複製するには大掛かりな器具が必要になることが想定される.以上の理由から,個人認証の要素に有効であると考えられるため頭部形状を用いる手法を提案する.\par
以降,\ref{related}章では関連研究を紹介する.\ref{method}章では提案手法,\ref{make}章では実装について述べる.\ref{evaluation}章では提案手法の評価実験の結果を述べつつ考察を行う.最後に\ref{conclude}章で本研究をまとめる.