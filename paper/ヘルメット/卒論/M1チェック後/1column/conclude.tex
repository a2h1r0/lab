\chapter{おわりに}
\label{conclude}
本研究では,圧力センサを内部に取り付けたヘルメットを着用することで頭部の形状を計測し,頭部形状の個人差から二輪車の所有者本人を識別する手法を提案した.評価実験を行うため,プロトタイプデバイスとデータ収集および解析プログラムを作成した.プロトタイプデバイスは市販のフルフェイス型ヘルメットを加工し,圧力センサを取り付けた.このプロトタイプデバイスからデータを取得するためにデータ収集プログラムを作成し,被験者9名から合計で360秒間の頭部形状データを取得した.解析に用いたプログラムはPythonで実装し,sklearn.covariance.MinCovDetでマハラノビス距離を計算,閾値を移動させて装着者の識別を行った.評価実験の結果,認証の精度の評価指標であるEERが個人ごとには約0\%$\sim$約10\%,全体では約7.8\%という結果が得られた.この結果より,本手法は個人識別手法として有効であると考えられる.今後はさらなるデータ収集を行い,実環境での提案手法の評価を行う.また,利用者のデータ群に差がないときの個人識別方法を定義し,検証していく.