\chapter{Conclusion}
\label{sec:conclusion}
In this thesis, I proposed a method that enables a PPG sensor to measure an arbitrary pulse wave by using a display. To investigate the effectiveness of the proposed method, I implemented display drawing programs and conducted evaluation experiments with five kinds of smartwatches and four kinds of displays. The results showed that the error between the target heart rate and the heart rate acquired by the smartwatch was within $3$ bpm in many cases when the target heart rate was set to 60--100 bpm. In contrast, when the target heart rate was set to 40--55 and 105--200 bpm, the measured heart rate could be input to the smartwatch with a small error under certain conditions. When the generated PPG data was imported into HRV analysis software, it was recognized as a pulse wave in the same way as real PPG data obtained from a person. By comparing the heart rate, RR interval, and SDNN calculated from the real and generated PPG data, I also confirmed that the proposed method could generate stable PPG data. On the other hand, when the waveforms were compared, the generated PPG waveform differed greatly from the real PPG waveform, which indicated that the software could calculate the heart rate, RR interval, SDNN, and LF/HF ratio regardless of the waveform. This result indicates that calculation of these parameters without verifying the waveform would be vulnerable to an attack with fake PPG data.\par

In my future work, I will improve the reproducibility of PPG data for use in a real environment and implement a mechanism that enables a wearable device on a display to measure the same PPG data by inputting PPG data obtained from a living body part. To achieve this, the system will need to automatically determine the colors to be drawn on the display while calibrating for the environment; I will thus build a generative model that can output the colors to be drawn from input PPG data.
