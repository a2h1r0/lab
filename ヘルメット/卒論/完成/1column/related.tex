\chapter{関連研究}
\label{related}
本章では個人認証手法,頭部装着型デバイス,頭部状態認識に関する研究を紹介する.
\section{個人認証手法}
%車両にカメラを取り付ける手法
当麻ら\cite{face}はステレオカメラを用いた顔認証システムを提案している.既存の顔認証システムではカメラに意識して顔を向ける必要があった.その問題を解決するため,少数の正面を向いていない顔画像から正面顔画像を作成し,個人認証を行う手法を提案している.佐藤ら\cite{door}は掌紋認証を装備したインテリジェントドアノブシステムの開発をしている.このシステムでは,認証に特別な動作を必要とする煩わしさを解決するため,ドアノブにカメラを装着し掌紋の画像を取得する.得られた画像からSIFT特徴を用いて認証を行う.

%ヘルメットでのジェスチャー認証
成ケ澤ら\cite{acceleration}はスマートフォン内蔵の加速度センサとジャイロセンサを用い,ユーザがスマートフォンを手のひらの上で回転させる動作を行うことで,その加速度と角速度の特徴量から認証する手法を提案している.

%ヘルメットに搭載可能な別認証手法
白川ら\cite{iris_eye}は虹彩と目の周辺画像を統合して認証する手法を提案している.虹彩認証は高画質の画像を必要とし,至近距離で認証を行う必要があり,被認証者に負担を与えてしまう.そこで目頭や目尻,まぶたの形などに個人差が存在することに注目し,目の周辺の分割画像を利用して目の周辺認証を行い,虹彩と統合する認証を提案した.岸里ら\cite{mouth_pattern}は口唇領域の動きの画像認識を用いたスマートデバイス向けパターンロックシステムを提案している.スマートフォンやタブレット端末のパターンロック認証は,肩越しに画面を覗き見るショルダーハックや,画面に残った脂の跡を読み取ることによって,認証キーを盗まれてしまうリスクをもつ.そこで手の指の代わりに,端末のカメラで口唇領域の動きを認識し,画面に触れることなく認証を行うシステムを提案した.越前ら\cite{finger_print}は写真からの指紋復元の脅威とその対策技術を提案している.指紋認証には複製の恐れがあり,少ない写真でも簡易に複製が可能である.そのため,指にジャミングパターンを装着する対策技術を提案した.

%上記の認証手法へのダメ出しと提案手法の強み
これらは個人認証の手法であり,いずれも二輪車のキーとして応用ができる可能性がある.顔認証を行う場合,カメラをあらかじめ車両に取り付けておき,ヘルメットを装着する前にカメラの方を向くことでシステムを実装することが可能である.掌紋画像での認証の場合は二輪車のハンドルにカメラを取り付けることで実装可能である.しかし,どちらの手法も悪天候による水没,走行時の衝撃からの耐久性を考慮しなければならない.また,認証時にカメラの電源を入れる手間が生じる.したがって,カメラを車両に取り付ける必要のある手法は向かない.ジェスチャ認証の場合,ヘルメットに加速度センサとジャイロセンサを搭載することで,ヘルメットを被るまでの動作の加速度と角速度の特徴量を用いて認証できる可能性がある.しかしながら,急いでいて動作が速くなる場合や,雨天時にヘルメットの内装が濡れないように気をつけながら装着する場合など,ヘルメットを装着する動作が変化する可能性が考えられる.状況の変化を考慮すると,ヘルメットを静止させたまま認証が可能な手法が適切である.

目を用いた認証手法の場合,ヘルメットを動かすことなく認証が可能である.しかし,虹彩や目の周辺の分割画像を取得するには目の前付近にカメラを設置する必要がある.そのため,ヘルメットに実装する場合,視界を遮る恐れがある.口唇領域の画像であれば視界を遮らずに撮影できるが,ヘルメットの口元の空間は限られる.そのため,口とカメラの距離が近くなってしまい,1個のカメラで口唇領域の動きを判別することは難しくなる.カメラの数を増やすとヘルメットの重量増加に繋がる.

そこで,本研究では内装に圧力センサを搭載したヘルメットを装着することで,頭部形状を取得して個人認証する手法を提案する.提案手法は個人認証のために特別な動作を必要とせず,デバイスの搭載によって視界を遮ることもない.さらに,認証に用いる頭部形状を複製するには立体形状を正確に把握する必要があり,複製が困難である.

\section{頭部装着型デバイス}
%デバイスとしての新規性
田中ら\cite{glasses}はメガネ型デバイスを用いた経皮水分蒸散量の常時測定システムを提案している.皮膚状態の診断には経皮水分蒸散量などの定量的な指標が用いられるが,測定には高価な機器が必要で,また定常的な測定はできない.そこで,定常的に測定を行い皮膚の健康維持を支援するため,メガネ型デバイスに2つの温度・湿度センサを装着し,皮膚状態の評価指標である経皮水分蒸散量の常時測定を行うシステムを提案した.石井ら\cite{happymouth}は人間の対面コミュニケーション能力を拡張するマスク型デバイス「HappyMouth」を提案している.このシステムでは,マスクに小型ディスプレイが内蔵されており,口元の映像を提示できる.映像提示の機能として,ユーザが自分の好みの口を選択して表示する機能,ユーザの発話をテキスト化して字幕表示する機能,ユーザの発したキーワードをインターネットで画像検索した結果を表示する機能がある.新島ら\cite{cap_sensor}は左右の側頭筋の筋活動を測定することができる,導電性高分子の布電極を用いた帽子型筋電センサhitoeCapを提案している.これは食事や睡眠や運動などの日々の生活の様々な場面で活動する,咀嚼筋の一つである側頭筋の筋電データを測定すれば,ユーザのライフログとして活用できると考え提案された.これらの研究はいずれも頭部に装着するデバイスであり,様々な形状のデバイスが提案されている.しかしながら,頭部装着型デバイスとしてヘルメットを用いた研究は筆者の知る限り存在しない.

\section{頭部状態の認識}
%頭部形状を認識するという点での新規性
近年注目されているヒアラブルデバイスにおいて求められる機能のひとつとして,手や視界を占有することのないデバイス操作機能が挙げられる.既存の製品や既存の研究では認識精度や認識できるジェスチャの種類,頑健性などの点で課題が残る.これらの課題を解決するために雨坂ら\cite{ear}は首,顎,顔の状態(頭部状態)にともなって外耳道が変形することに着目し,外耳道インパルス応答を測定することで頭部状態を認識する手法を提案した.一方,状況の変化に関して口周辺の動作に着目すると,感情や咀嚼といった様々な情報が含まれており,これらを認識することで感情の記録や,咀嚼カウントによる肥満防止など,新たなコンテキストアウェアサービスが提供できる.そこで,山下ら\cite{mouth}は日常生活での利用を想定し,安価な付け髭型デバイスを用いた口周辺の形状変化によるコンテキスト認識手法を提案した.これらの研究は表情や動作などの動的な情報を取得するものであり,対して本研究は静的な頭部形状そのものの特徴を取得するという点で異なる.