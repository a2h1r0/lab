\chapter{おわりに}
\label{conclude}
本研究では,圧力センサを内部に取り付けたヘルメットを装着することで頭部の形状を計測し,頭部形状の個人差から二輪車の所有者本人を認証する手法を提案し,プロトタイプデバイスの実装とデータ収集および解析プログラムを作成した.プロトタイプデバイスは市販のフルフェイス型ヘルメットを加工し,圧力センサを取り付けた.実験では,プロトタイプデバイスからデータを取得するためにデータ収集プログラムを作成し,頭部形状データとして被験者9人からそれぞれ2秒間のセンサ値を20回分取得した.解析に用いたプログラムはPythonで実装し,登録データと入力データのマハラノビス距離を計算し,本人認証するか拒否するかの閾値を移動させて認証性能を評価した.評価実験の結果,認証の精度の評価指標であるEERが9名中4名が0.012以下,平均0.076という結果が得られた.この結果より,本手法は本人認証手法として有効であると考えられる.今後はさらなるデータの収集を行い,実環境での提案手法の評価を行う.