% \documentclass[Japanese]{dicomopapers}
\documentclass[Japanese,noauthor]{dicomopapers}
\usepackage[dvips]{graphicx}
\usepackage{latexsym}

\def\Underline{\setbox0\hbox\bgroup\let\\\endUnderline}
\def\endUnderline{\vphantom{y}\egroup\smash{\underline{\box0}}\\}
\def\|{\verb|}
\def\newblock{\hskip .11em plus .33em minus .07em}

%概要投稿用余白調整ここから
\setlength{\Jauthorjreceivesep}{0.0mm}
\setlength{\Jreceivejabstsep}{0.0mm}
\setlength{\Jabstsepjkeyword}{0.0mm}
\setlength{\Jkeywordetitle}{0.0mm}
%概要投稿用余白調整ここまで

\begin{document}

% 和文表題
\title{ディスプレイを用いた脈波生成手法}

% 英文表題
\etitle{How to Typeset Your Abstract in {\LaTeX}}

% 所属ラベルの定義
\affiliate{Rits}{立命館大学大学院情報理工学研究科}
\affiliate{JST}{JSTさきがけ}

\author{藤井 敦寛}{Atsuhiro Fujii}{Rits}
\author{村尾 和哉}{Kazuya Murao}{Rits, JST}

% 表題などの出力
\maketitle

% 本文はここから始まる
\section{研究の背景と目的}
健康管理への意識の高まりから,自身の生体情報を記録するウェアラブルデバイスが広く普及している.記録する生体情報は活動量や呼吸数,体温などさまざまな情報があり,心拍数もそのひとつである.心拍数を取得するために用いられる脈波センサは,緑色のLEDを皮膚に照射して,血管を通して反射した光の変化から脈波を計測する光電式容積脈波記録法(PPG)と呼ばれる方式のものが一般的であり,市販のスマートウォッチに導入されている.スマートウォッチのセンサから取得できる心拍データを用いて疲労度を検出する手法を今井ら\cite{fatigue_detection}が提案しているなど,脈波センサから得られるデータを使用した研究は盛んである.
\par

しかしながら,義手やロボットアームなど人工的な身体には血流が存在しないため,スマートウォッチを生身の身体と同様に手首に装着して生体情報を計測することはできない.この問題を解決するには,現状では追加のセンサを計測可能な身体部位に装着して,何からの通信手段を用いてセンサデータを収集する必要がある.しかし,仮に収集したとしても,スマートウォッチが提供するアプリにデータを与えてサービスを利用することはハードルが高いため,スマートウォッチに搭載されているセンサに計測させたい.
\par

本研究では,ディスプレイを用いて脈波センサに脈波データを計測させる手法を検討する.ディスプレイの表示を変化させることで任意の脈波データを脈波センサに読み取らせることができれば,義手やロボットアームなどの人工的な身体にスマートウォッチを装着する場合でも,身体と義手の接合部などで計測された脈波を入力することで,その値をスマートウォッチに読み取らせることができ,スマートウォッチが提供する機能を生身の身体と同様に利用できる.本稿ではあらかじめ収集された実際の脈波データを参考にして,ディスプレイの色調を変化させることで,任意の心拍数を計測させる手法を提案する.


\section{予備実験}
20代男性1名の左手人差し指に光電式容積脈波記録法の脈波センサ(pulsesensor.com製)を装着し,サンプリング周波数約90Hzで10秒間,参考にする脈波データを収集した.
\par

擬似脈波の生成には,データの収集で使用するPCとは異なるPCのディスプレイを使用した.ディスプレイ上に脈波センサを乗せ,光が入らないように布で覆った後,ガムテープで固定した.参考にするための脈波データを取得した時と同じ条件でデータの取得を行った.ディスプレイの色調の変化にはJavaScriptを使用し,ブラウザの背景色を変化させることで制御した.参考にする脈波データを1サンプルずつ読み込み,その値に応じた3色で表示を繰り返す.全サンプルの処理が終了した場合,同じデータで再び処理を行う.また,色の表示ごとに10[ms]の遅延を挟んだ.
\par

データの取得結果から,擬似的にピークを生成できていることが確認できた.したがって,ディスプレイを使用するアプローチは有効だといえる.


\section{まとめと今後}
本研究では,ディスプレイを用いて擬似的に脈波データを生成する手法を実現するために,ディスプレイの色調を変化させることで,脈波センサの取得値を意図的に操作することが可能であるか調査した.今後は,別の身体部位から取得された脈波データをリアルタイムに再現するプログラムを実装する.そのためには,手動ではなく自動でディスプレイの色調を決定し,変化に適応していく必要があるため,何らかの機械学習モデルを構築することを検討している.


\bibliography{references}
\bibliographystyle{junsrt}

\end{document}
