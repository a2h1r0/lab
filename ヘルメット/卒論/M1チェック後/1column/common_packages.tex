%-----------------------------------------------------------------%
\makeatletter% @付きのコマンドが使えるようにする。
%ファイル末尾で \makeatother で最後で解除すること
%-----------------------------------------------------------------%

% パッケージ読み込み部:開始
% 有用そうなのをいくつか列挙しておきます.
% 他にも様々あります.
%   参考URL:http://www.biwako.shiga-u.ac.jp/sensei/kumazawa/texindex.html

% 画像挿入用パッケージ: graphicx
% 文字等に色づけする場合に必要: color
% 
%   画像の利用方法 (png, jpg/jpeg, pdf、などが利用できる)
%   例 
%     \begin{figure}
%       \includegraphics[width=10cm,bb=0 0 600 600]{image/sample.png}
%     \end{figure}
%
%   注
%     []にて指定している値は,[表示横幅, bb=0 0 画像横pixel数 画像縦pixel数]です.
%     bb=以後は,ebbコマンドなどでsample.bbを作成しておけば不要です.
%     Windows : makebb.bat
%     @echo on
%      for %%i in (*.png)  do ebb %%i
%      for %%i in (*.pdf)  do ebb %%i
%      for %%i in (*.jpg)  do ebb %%i
%      for %%i in (*.jpeg) do ebb %%i
%
%     Linux / MacOSX : makebb.sh
%      #!/bin/sh -e
%      cd `dirname "$0"`
%      xbb *.pdf *.png *.jpg *.jpeg && ls *xbb | \
%      perl -ne 'chomp; ~s/\.xbb//g; print "mv $_.xbb $_.bb\n"' | \
%      sh 
%
% =======
%
\usepackage[dvipdfmx]{graphicx, color}

% 背景にすかし文字 (DRAFT) を表示 : pdfdraftcopy : graphicx に依存
%   提出用と間違えるのを予防
%   開発元URL: http://sarovar.org/projects/pdfdraftcopy/
%
% IsDraft が \def されている場合だけ有効
\@ifundefined{IsDraft}{}{\usepackage[watermark]{pdfdraftcopy}
\watermarkgraphic{images/draft_copy.pdf}}

% ブロックコメント
% \begin{comment} \end{comment}
\usepackage{comment}

% しおりの作成、目次や参考文献をクリッカブルリンクへ
% いろいろなパッケージと互換性問題があるようなので,他のパッケージを利用する場合にはチェックすること
% 依存関係情報: http://oku.edu.mie-u.ac.jp/~okumura/texwiki/?hyperref#z442922c
%
\usepackage[dvipdfm
	, bookmarks=true
	, bookmarksopen=true
 	, bookmarksnumbered=true
	, CJKbookmarks=true
%
	, backref=true
	, pdfborder={0 0 0} % width
%
	, colorlinks=false
	, frenchlinks=false
%	, linkbordercolor={1 1 1} % RGB
%	, citebordercolor={1 1 1} %RGB
%	, urlbordercolor={1 1 1} %RGB
%	, pdfhighlight=/I
]{hyperref}
% しおりの文字化け解消.必要な文字コード指定を入れる.
%\AtBeginDvi{\special{pdf:tounicode 90ms-RKSJ-UCS2}}% SJIS 
%\AtBeginDvi{\special{pdf:tounicode EUC-UCS2}} % EUC
%\AtBeginDvi{\special{pdf:tounicode UTF8-UCS2}} % UTF-8
%
% 文字コードを自動判定してしおりの文字化けを解消できるパッケージ.
% 上のオプションをオフにしてこちらを使ってもよい.
\usepackage{pxjahyper}
%
% PDFのメタ情報を設定したければ以下を追加
%\hypersetup{
%	pdftitle={タイトル},
%	pdfsubject={小見出し},
%	pdfauthor={PDF制作者名},
%	pdfkeywords={キーワード1; キーワード2; キーワード3;},
%}

% urlをそのまま表記可能にする.
%   本来であれば><%#^~などはエスケープしないといけない
%   また、1 word として認識されるために、自動で折り返してくれないURLを、
%   途中で折り返しをしてくれる。
%
% 使い方:
%    \url{http://www.ubi.cs.ritsumei.ac.jp}
%    \url{<a href=``http://www.ubi.cs.ritsumei.ac.jp''>Link</a>}
\usepackage{url}
\urlstyle{rm}
% 数式パッケージ
%   美文書作成などを参照してください.
%\usepackage{amsmath, amssymb}

% 表用パッケージ
%   きれいな表が書けます.
%   参考URL: http://www.biwako.shiga-u.ac.jp/sensei/kumazawa/tex/hhline.html
\usepackage{hhline, multirow}

% 表の見栄えを良くする
% 表の架線が上下だけ太くなります.
%   参考URL: http://www.biwako.shiga-u.ac.jp/sensei/kumazawa/tex/booktabs.html
\usepackage{booktabs}

% いろいろな囲み線作成用
%   様々な場面で利用できます.便利です.
%   参考URL: http://www.biwako.shiga-u.ac.jp/sensei/kumazawa/tex/fancybox.html 
%   参考URL: http://www.biwako.shiga-u.ac.jp/sensei/kumazawa/tex/ascmac.html
\usepackage{fancybox, ascmac}

% 可換図を書けるように
%   参考URL: http://www.biwako.shiga-u.ac.jp/sensei/kumazawa/tex/amscd.html
% \usepackage{amscd}

%% 以下、私作のコマンド
\usepackage{here}
%
% \TS{hoehoge}: パラグラフのトピックセンテンスを指定
% \listofts: トピックセンテンスを抽出してリスト化: 目次のような感じ
%
% \TODO{}: TODO を出力 (赤色で TODO: hogehoge)
% \listoftodo: TODOを抽出してリスト化: 目次のような感じ
%
%%%%%%%%%%%%%%%%%%%%%%%%%%%%%%%%%%%%%%%%%%%%%
%
\def\RED#1{{\color{red}#1}}%
%
\def\TS#1{\AddToContLine{lots}{#1}#1}%
\newcommand{\listtsname}{Paragraph Topic Sentenses (\RED{for Debug})}%
\newcommand{\listofts}{\listof{ts}}%
%
\newcommand{\TODO}{\secdef\@TODO\@sTODO}%
\def\@TODO[#1]#2{\@sTODO{#2}\@ifundefined{PrintTodo}{}{\\}}%
\def\@sTODO#1{\AddToContLine{lotodo}{#1}\@ifundefined{PrintTodo}{}{\RED{TODO: #1}}}% started TODO means [ \TODO*{hogehoge} ]
\newcommand{\listtodoname}{List of TODO (\RED{for Debug})}%
\newcommand{\listoftodo}{\@ifundefined{PrintTodo}{}{\listof{todo}}}%
\newcommand{\forceListoftodo}{\listof{todo}}%
%
%
%%%%%%%%%%%%%%%%%%%%%%%%%
\def\listof#1{%
  \if@twocolumn\@restonecoltrue\onecolumn%
  \else\@restonecolfalse\fi%
  \abstract{{\csname list#1name\endcsname}}%
  \@mkboth{{\csname list#1name\endcsname}}{}%
  \@starttoc{lo#1}%
  \if@restonecol\twocolumn\fi%
}%
\def\AddToContLine#1#2{%
\@ifundefined{c@atcl#1}{\newcounter{atcl#1}[chapter]}{}%
\ifnum \csname c@atcl#1\endcsname =0 % chapter用のダミーを追加
  \refstepcounter{atcl#1}%
  \addcontentsline{#1}{chapter}{\protect\numberline{\@chapapp\thechapter\@chappos}}%
\fi%
\ifnum \c@section>0 %
  \ifnum \c@subsection >0 %
    \ifnum \c@paragraph >0 %
      \ifnum \c@subparagraph >0 %
        \AddToContLineSec{#1}{subparagraph}{#2}%
      \else\AddToContLineSec{#1}{paragraph}{#2}\fi%
    \else\AddToContLineSec{#1}{subsection}{#2}\fi%
  \else\AddToContLineSec{#1}{section}{#2}\fi%
\else%
  \ifnum \csname c@atcl#1\endcsname >0 %
    \addcontentsline{#1}{section}{\protect\numberline{}#2}%
  \else%
    \refstepcounter{atcl#1}%
    \addcontentsline{#1}{chapter}{\protect\numberline{\@chapapp\thechapter\@chappos}#2}%
  \fi%
\fi%
}%
\def\AddToContLineSec#1#2#3{%
  \@ifundefined{c@atcl#2#1}{\newcounter{atcl#2#1}[#2]}{}
  \ifnum \csname c@atcl#2#1\endcsname >0 %  各セクションの一番始めで無ければ章番号無し
    \addcontentsline{#1}{#2}{\protect\numberline{}#3}%
  \else%
    \refstepcounter{atcl#2#1}%  セクションの一番始めであれば章番号付き
    \addcontentsline{#1}{#2}{\protect\numberline{\csname the#2\endcsname}#3}%
  \fi%
}%
%
\def\BoldFirstRef#1#2{\@ifundefined{c@#1}{\newcounter{#1}\textbf{#2}}{#2}}%
\@ifundefined{figref}{\def\figref#1{\BoldFirstRef{figref#1}{図 \ref{#1}}}}{}% 図出力
\@ifundefined{tabref}{\def\tabref#1{\BoldFirstRef{tabref#1}{表 \ref{#1}}}}{}% 表出力
\@ifundefined{secref}{\def\secref#1{\ref{#1} 節}}{}%
\@ifundefined{chapref}{\def\chapref#1{第 \ref{#1} 章}}{}%
\def\SetJsDescriptionIndent#1{%
%%%%%%%%%%%%%%%%%%%%%%%%%%%%%
\renewenvironment{description}{%
  \list{}{%
    \leftmargin=#1
    \labelwidth=\leftmargin
    \labelsep=1zw
    \advance \labelwidth by -\labelsep
    \let \makelabel=\descriptionlabel}}{\endlist}
%%%%%%%%%%%%%%%%%%%%%%%%%%%%%
}%
\def\abstract#1{%
\chapter*{#1}\@mkboth{#1}{}%
\@ifundefined{c@abstract}{\newcounter{abstract}\pagenumbering{roman}}{}%
}%
\def\startMain{%
\if@openright\cleardoublepage\else\clearpage\fi%
\pagenumbering{arabic}%
}
%-----------------------------------------------------------------%
%-----------------------------------------------------------------%
%
%
\def\BachelorThesis{卒 業 論 文}
\def\MasterThesis{修 士 論 文}
\def\DoctoralThesis{博 士 論 文}
\def\@thesisTitle{圧力センサ搭載ヘルメットを用いた\\個人識別手法の提案}
\def\@thesisSupervisor{卒業研究3 (1O):村尾 和哉 担当} % 卒業研究3のクラス名と担当教員名を記載
%\def\@thesisSupervisor{卒業研究3 (1H):西尾 信彦 担当}
\def\@thesisKind{\MasterThesis}
\def\@thesisYear{2019年}
%\def\@thesisYear{2019年}
%\def\@thesisEraName{平成XX年}
\def\@thesisSemester{秋学期} % 秋か春を記載
%\def\@thesisSemester{秋学期}
\def\@thesisAuther{藤井 敦寛}
\def\@thesisNameOfGraduateSchool{立命館大学大学院}
\def\@thesisNameOfUnderGraduate{立命館大学 情報理工学部}
\def\@thesisNameOfMajor{情報理工学研究科情報理工学専攻}
\def\@thesisNameOfGrade{情報システム学科 4回生 26001603573} % 2017年度入学者はコース名に変更
%\def\@thesisNameOfGrade{情報システム学科 4回生 2600xxxxxxxx}
\def\thesisTitle#1{\gdef\@thesisTitle{#1}}
\def\thesisKind#1{\gdef\@thesisKind{#1}}
\def\thesisYear#1{\gdef\@thesisYear{#1}}
\def\thesisEraName#1{\gdef\@thesisEraName{#1}}
\def\thesisAuther#1{\gdef\@thesisAuther{#1}}
\def\thesisNameOfGraduateSchool#1{\gdef\@thesisNameOfGraduateSchool{#1}}
\def\thesisNameOfMajor#1{\gdef\@thesisNameOfMajor{#1}}
\def\makeThesisTitle{
\begin{titlepage}%
\centering
\begin{minipage}[]{\fullwidth}
\large
 

% 表紙の配置や文字のサイズは以下で調整
\vspace{6mm}
{\@thesisYear}度({\@thesisSemester})\\
{\@thesisSupervisor}\\%
\begin{center}
\vspace{2mm}
{\Large {\@thesisKind}}\\
\vspace{27mm}
%
{\LARGE {\@thesisTitle}}\\%
\vspace{84mm}%
{\@thesisNameOfUnderGraduate}\\%
%{\@thesisNameOfGraduateSchool}\\%
{\@thesisNameOfGrade}\\%
%{\@thesisNameOfMajor}\\%
%
\vspace{7mm}%
{\Large {\@thesisAuther}}%
\end{center}%
\end{minipage}%
\end{titlepage}%
}
%
%-----------------------------------------------------------------%
\makeatother%
%-----------------------------------------------------------------%
