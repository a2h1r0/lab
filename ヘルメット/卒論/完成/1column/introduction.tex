\chapter{はじめに}
\label{introduction}
現在販売されている四輪車の多くにスマートキーシステムが導入されている.スマートキーシステムとは,キーをポケットなどに入れて接近するとキーから発せられる電波を車が感知してドアの解錠を行い,離れるとドアの施錠を行う機能である.キーを持ったままボタンを押すだけでエンジンを始動することもできる.スマートキーシステムは二輪車にも導入されつつある.二輪車におけるスマートキーシステムも四輪車と同様に,キーを所持しておくだけでキーシリンダーにキーを挿し込む手間なく,ラゲッジスペースと呼ばれる荷物を収納するスペースの解錠やエンジンの始動を可能にする機能である.しかしながら,スマートキーシステムはキーを所持しておかなければならず,キーの紛失や第三者がキーを入手することで車両盗難のリスクがある.\par

物理的な鍵以外で本人を認証する手法として,パスワードなどの知識や筆跡などの行動的特徴,指紋などの身体的特徴を用いる手法がある.当麻ら\cite{face}はステレオカメラを用いた顔認証システムを提案しており,ヘルメット装着前に二輪車に取り付けられたカメラの方を向くことで実現可能であると考えられる.しかし,カメラの電源を入れてカメラの方を向く必要があり,二輪車を運転するたびにこのような操作を行うのは面倒である.また,本人の写真やお面で認証を破ることができるリスクもある.このほか,指紋認証のアプローチもあるが,指紋は写真などから簡単に複製されるリスク\cite{finger_print}がある.\par

本研究では,二輪車での走行において法令で装着が義務付けられているヘルメットに着目し,ヘルメットを用いた本人認証が実現できれば,既存のキーを用いたシステムの問題点を解決できると考え,搭乗者が圧力センサを搭載したヘルメットを装着することで,搭乗者の頭部形状にもとづき本人認証をする手法を提案する.頭部形状は個人ごとに異なり,頭部形状の取得や複製は困難であると考える.具体的には,あらかじめ車両とペアリングして所有者の頭部形状データを登録しておいたヘルメットを二輪車のヘルメットロックに装備しておく.ヘルメットロックに繋がったまま利用者がヘルメットを装着し,取得した頭部形状と事前に登録した頭部形状が一致すればロックが外れ,エンジンを始動できるようになるという想定である.\par

提案手法によって利用者は何も持たずに乗車できるため,キーの紛失や盗難のリスクを減少させることができる.一方で,登録された利用者以外がヘルメットをかぶった場合は本人として認証されず,ヘルメットロックも外れず,エンジンの始動もできないため,車両盗難のリスクも減少させることができる.\par

以降本稿では,\ref{related}章で関連研究を紹介する.\ref{method}章で提案手法を説明し,\ref{make}章で実装について述べる.\ref{evaluation}章で提案手法の評価実験と結果の考察を行い,最後に\ref{conclude}章で本研究をまとめる.