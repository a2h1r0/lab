\documentclass[English]{dicomopapers}

\usepackage[dvips]{graphicx}
\usepackage{latexsym}

\def\Underline{\setbox0\hbox\bgroup\let\\\endUnderline}
\def\endUnderline{\vphantom{y}\egroup\smash{\underline{\box0}}\\}
\def\|{\verb|}


\setlength{\Etitleauthorsep}{5.0mm}
\setlength{\Eauthorreceivesep}{0.0mm}
\setlength{\Eabstkeywordsep}{0.0mm}

\begin{document}

\title{How to Typeset Your Abstract in {\LaTeX}}

\affiliate{IPSJ}{IPSJ}
\affiliate{DICOMO}{DICOMO2020}

\author{TARO JOHO}{IPSJ}
\author{HANAKO SHORI}{DICOMO}

\maketitle

\section{About DICOMO abstract template}
This file is a {\LaTeX} template for abstract writing, based on
DICOMO2020 full paper format. With this file you are expected to
submit a 1 page abstract for DICOMO2020 application.

\section{Notices for abstract writing}
\begin{itemize}
 \item Abstract volume
   
   The abstract must be contained in 1 page, including title, author,
   affiliation and summary of the work.

 \item Chapters
     
   There is no need to provide multiple chapters. One chapter for an
   abstract is enough. Use of paragraphs, itemization that make the
   content readable are recommended.

 \item Figures and tables

   You can add figures and/or tables to clarify the advantages of your work.

 \item References

  You can add references.

 \item Relationship with camera-ready submission

   This abstract is not published. You can use this abstract as a
   draft for your camera-ready submission.

 \item Bibliographic information alignment

   This abstract's bibliographic information submitted at the same
   time (title, author, affiliation, keywords etc.) is used for
   DICOMO2020 session arrangement and publication. Keep integrity of
   the information to avoid mistakes.
\end{itemize}

\end{document}
