\chapter{Introduction}
\label{sec:introduction}
With the growing awareness of health management, wearable devices that record biometric information have become widely used. The recorded biometric information includes a variety of data, such as activity, respiration, body temperature, cardiac potential, blood pressure, gaze, pulse wave, and heart rate. The pulse sensor used to acquire the latter two kinds of data (i.e., the pulse wave and heart rate) irradiates the skin with LEDs that emit infrared light, red light, or green light with a wavelength around 550 nm. The oxidized hemoglobin in the blood flowing through arteries can absorb these lights. The pulse sensor takes advantage of the fact that the amount of reflected light decreases as the arterial blood flow increases with the timing of a heartbeat. Specifically, it uses a phototransistor to obtain changes in the amount of reflected light and measure the pulse wave. Here, the pulse wave data is numerical data on the changes in the reflected light, and the heart rate is measured by detecting the peaks that appear in the pulse wave data. This type of pulse wave measurement technique is called photoplethysmography (PPG) \cite{ppg}. Today's PPG sensors use the same principle as the device originally introduced by Hertzman \cite{ppg_principle1, ppg_principle2}, but in a smaller size using modern equipment. Many commercially available wearable devices (e.g., smartwatches) are equipped with a PPG sensor as a pulse sensor.\par

I speculate that it may be possible to measure an arbitrary pulse wave by inputting a change of light to a PPG sensor. Accordingly, I propose a method that enables a PPG sensor to acquire pulse data from a display. To simplify data collection for photoplethysmography imaging (PPGI), Paul et al. \cite{ppg_generator} developed a hardware PPG simulator by using an LED array to generate PPG signals. My method differs in that it aims to use a small display device to input data to the PPG sensor of a wearable device. Specifically, my method inputs an arbitrary heart rate to a smartwatch. I set two objectives for the proposed method: PPG transfer and recognition of fake PPG data.\par

In terms of PPG transfer, artificial bodies and parts such as prosthetic hands, robotic arms, and telepresence robots do not have any blood flow, which makes it impossible to measure biometric data even if a smartwatch is worn on the wrist. While typical smartwatch functions such as calling, messaging, clocks, and payments, as well as sensors such as accelerometers and GPS sensors, can still be used with artificial limbs as with living limbs, pulse data cannot be measured. Meanwhile, when a smartwatch is attached to other body parts where blood flow exists (e.g., an ankle) to measure pulse data, the usability of other functions (e.g., messaging) is reduced. Other possible methods for PPG transfer include attachment of an additional PPG sensor to other body parts with blood flow and wireless input of the PPG data to the smartwatch, or detection of PPG data (or heart rate data) by non-PPG sensors \cite{heart_rate_accelerometer, Biowatch, SeismoTracker, heart_rate_ecg, heart_rate_touchscreen}. However, because most publicly available applications that use PPG data read the data from PPG sensors included in a device, the PPG data collected by these methods may not be usable for many applications.\par

In contrast, with the proposed method, even when a smartwatch is attached to an artificial limb, the user's pulse data can be read by changing the light of the display under the smartwatch's PPG sensor in accordance with pulse data measured at the junction of the living and artificial limbs. It is thus possible to use the normal functions of the smartwatch, because it is not modified and only the display is mounted on the artificial limb. Accordingly, users can still compare various aspects of commercial smartwatches, such as the design, function, and weight, and use the model of their choice. In addition, the smartwatch's PPG sensor can acquire data without modification of the smartwatch, which allows the user to still use common applications. Furthermore, in the case of a remote robot avatar, the operator's biometric data can be measured on the avatar's body.\par

As for recognition of fake PPG data, if a PPG sensor measures an arbitrary heart rate by the proposed method, it might be possible for a malicious user to falsify the heart rate and pretend to be exercising or continuing to rest. If a device using the proposed method becomes widely feasible and has a significant social impact, it will be necessary to examine the use of current PPG sensors in terms of this vulnerability.\par

In the rest of the thesis, I introduce related works in Chapter \ref{sec:related}. I then explain the details of the proposed method in Chapter \ref{sec:method} and evaluate it in Chapter \ref{sec:evaluation}. Finally, Chapter \ref{sec:limitation} describes my method's limitations, and Chapter \ref{sec:conclusion} concludes the thesis.
